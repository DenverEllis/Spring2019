% Created 2020-09-03 Thu 23:08
% Intended LaTeX compiler: pdflatex
\documentclass[11pt]{article}
\usepackage[utf8]{inputenc}
\usepackage[T1]{fontenc}
\usepackage{graphicx}
\usepackage{grffile}
\usepackage{longtable}
\usepackage{wrapfig}
\usepackage{rotating}
\usepackage[normalem]{ulem}
\usepackage{amsmath}
\usepackage{textcomp}
\usepackage{amssymb}
\usepackage{capt-of}
\usepackage{hyperref}
\author{Denver Ellis}
\date{\textit{<2020-08-31 Mon>}}
\title{Assignment01: Matrix Assignment}
\hypersetup{
 pdfauthor={Denver Ellis},
 pdftitle={Assignment01: Matrix Assignment},
 pdfkeywords={},
 pdfsubject={},
 pdfcreator={Emacs 27.1 (Org mode 9.3)}, 
 pdflang={English}}
\begin{document}

\maketitle
\tableofcontents


\section{Vector}
\label{sec:org6771878}
\subsection{Exercise 1}
\label{sec:orgcf8f526}
\subsubsection{Question}
\label{sec:org5592d7b}
Consider two vectors, a, b

\begin{verbatim}
> a <- c(1,5,4,3,6)
> b <- c(3,5,2,1,9)
\end{verbatim}

What is the value of: a <= b

\subsubsection{Answer}
\label{sec:org85d8f69}
\begin{verbatim}
> a <= b
[1]  TRUE  TRUE FALSE FALSE  TRUE
\end{verbatim}

\subsection{Exercise 2}
\label{sec:org54b8fd1}
\subsubsection{Question}
\label{sec:orgf08ae05}
Consider two vectors, x, y

\begin{verbatim}
> x <- c(12:4)
> y <- c(0,1,2,0,1,2,0,1,2)
\end{verbatim}

What is the value of: \texttt{which(!is.finite(x/y))}

\subsubsection{Answer}
\label{sec:orge55e521}
\begin{verbatim}
> which(!is.finite(x/y))
[1] 1 4 7
\end{verbatim}

\subsection{Exercise 3}
\label{sec:orgaca3d38}
\subsubsection{Question}
\label{sec:org8a74de2}
\begin{verbatim}
> x <- c(1,2,3,4)
\end{verbatim}

What is the value of k for:
\begin{verbatim}
> (x+2)[(!is.na(x)) & x > 0] -> k
\end{verbatim}

\subsubsection{Answer}
\label{sec:org38f6513}
\texttt{k} stores the output of the evaluated statement.

\subsection{Exercise 4}
\label{sec:orgb25b1d9}
\subsubsection{Question}
\label{sec:org3fd77dd}
If
\begin{verbatim}
> x <- c(2, 4, 6, 8)
> y <- c(TRUE, TRUE, FALSE, TRUE)
\end{verbatim}

What is the value of: \texttt{sum(x[y])}

\subsubsection{Answer}
\label{sec:org571159d}
The value of \texttt{sum(x[y])} is 14

\subsection{Exercise 5}
\label{sec:orgfc28f85}
\subsubsection{Question}
\label{sec:org5de2a13}
Consider the vector:

\begin{verbatim}
> x <- c(34, 56, 55, 87, NA, 4, 77, NA, 21, NA, 39)
\end{verbatim}

Which R-statement will count the number of NA values in x?

\begin{center}
\begin{tabular}{ll}
a & count(is.na(X))\\
b & length(is.na(x))\\
c & sum(is.na(x))\\
d & count(!is.na(x))\\
e & sum(!is.na(x))\\
\end{tabular}
\end{center}

\subsubsection{Answer}
\label{sec:orgf7474d2}
The answer is \texttt{c: sum(is.na(x))}

\section{List}
\label{sec:org8ef270f}
\subsection{Exercise 1}
\label{sec:org60d1f90}
\subsubsection{Question}
\label{sec:org5cefb46}
If:
\begin{verbatim}
> p <- c(2,7,8)
> q <- c("A", "B", "C")
> x <- list(p,q)
\end{verbatim}

Then what is the value of \texttt{x[2]}:
\begin{center}
\begin{tabular}{ll}
a & NULL\\
b & "A" "B" "C"\\
c & "7"\\
\end{tabular}
\end{center}

\subsubsection{Answer}
\label{sec:orga86f3b3}
The answer is \texttt{b: "A" "B" "C"}

\subsection{Exercise 2}
\label{sec:org50552bd}
\subsubsection{Question}
\label{sec:org1924a75}
If:
\begin{verbatim}
> w <- c(2,7,8)
> v <- c("A", "B", "C")
> x <- list(w,v)
\end{verbatim}

Then write an R statement tha will replace "A" in x with "K". The exepected output:
\begin{verbatim}
> x
[[1]]
[1] 2 7 8

[[2]]
[1] "K" "B" "C"
\end{verbatim}

\subsubsection{Answer}
\label{sec:org6edf46c}
\begin{verbatim}
> x[[2]] <- c("K", "B", "C")
> x
[[1]]
[1] 2 7 8

[[2]]
[1] "K" "B" "C"
\end{verbatim}

\subsection{Exercise 3}
\label{sec:orgdc283b3}
\subsubsection{Question}
\label{sec:orgf72e5fc}
If:
\begin{verbatim}
> a <- list ("x"=5, "y"=10, "z"=15)
\end{verbatim}

Which R statement will give the sum of all elements in a?
\begin{center}
\begin{tabular}{ll}
a & sum(a)\\
b & sum(list(a))\\
c & sum(unlist(a))\\
\end{tabular}
\end{center}

\subsubsection{Answer}
\label{sec:orgec54415}
The answer is \texttt{c: sum(unlist(a))} with a value of 30

\subsection{Exercise 4}
\label{sec:orga75d902}
\subsubsection{Question}
\label{sec:orgc5d7393}
If:
\begin{verbatim}
> Newlist <- list(a=1:10, b="Good morning", c="Hi")
\end{verbatim}

Write an R statement that will add 1 to each element of the first vector in Newlist.  The expected output:
\begin{verbatim}
 > Newlist
$a
[1]  2  3  4  5  6  7  8  9 10 11

$b
[1] "Good morning"

$c
[1] "Hi"
\end{verbatim}

\subsubsection{Answer}
\label{sec:org8cad864}
\begin{verbatim}
> Newlist$a <- Newlist$a + 1
> Newlist
$a
 [1]  2  3  4  5  6  7  8  9 10 11

$b
[1] "Good morning"

$c
[1] "Hi"
\end{verbatim}

\subsection{Exercise 5}
\label{sec:org7cee7c2}
\subsubsection{Question}
\label{sec:org82cb37e}
If:
\begin{verbatim}
> b <- list(a=1:10, c="Hello", d="AA")
\end{verbatim}

Write an R expression that will give all elements, except the second, of the first vector of b.  The expected output:
\begin{verbatim}
[1]  1  3  4  5  6  7  8  9 10
\end{verbatim}

\subsubsection{Answer}
\label{sec:org51245df}
\begin{verbatim}
> which(b$a != b$a[2])
[1]  1  3  4  5  6  7  8  9 10
\end{verbatim}

\subsection{Exercise 6}
\label{sec:orgeb00b28}
\subsubsection{Question}
\label{sec:org29a7130}
Let
\begin{verbatim}
> x <- list(a=5:10, c="Hello", d="AA")
\end{verbatim}

Write an R statement to add a new item z = “NewItem” to the list x.  The expected output:
\begin{verbatim}
> x
$a
[1]  5  6  7  8  9 10

$c
[1] "Hello"

$d
[1] "AA"

$z
[1] "New Item"
\end{verbatim}

\subsubsection{Answer}
\label{sec:org74cfa45}
\begin{verbatim}
> x$z <- "NewItem"
> x
$a
[1]  5  6  7  8  9 10

$c
[1] "Hello"

$d
[1] "AA"

$z
[1] "NewItem"
\end{verbatim}

\subsection{Exercise 7}
\label{sec:org28a396f}
\subsubsection{Question}
\label{sec:org99bc27a}
Consider
\begin{verbatim}
> y <- list("a", "b", "c")
\end{verbatim}

Write an R statement that will assign new names ”one”, ”two” and ”three” to the elements of y.  The expected output:
\begin{verbatim}
> y
$one
[1] "a"

$two
[1] "b"

$three
[1] "c"
\end{verbatim}

\subsubsection{Answer}
\label{sec:orgcbfd405}
\begin{verbatim}
> names(y) <- c("one", "two", "three")
> y
$one
[1] "a"

$two
[1] "b"

$three
[1] "c"
\end{verbatim}

\subsection{Exercise 8}
\label{sec:orgb25cfd2}
\subsubsection{Question}
\label{sec:orgb34a74e}
If
\begin{verbatim}
> x <- list(y=1:10, t="Hello", f="TT", r=5:20)
\end{verbatim}

Write an R statement that will give the length of vector r of x.  The expected output:
\begin{verbatim}
[1] 16
\end{verbatim}

\subsubsection{Answer}
\label{sec:org3305e38}
\begin{verbatim}
> length(x$r)
[1] 16
\end{verbatim}

\subsection{Exercise 9 (Bonus)}
\label{sec:orgdae3e84}
\subsubsection{Question}
\label{sec:org53a749c}
Let
\begin{verbatim}
> string <- "Grand Opening"
\end{verbatim}

Write an R statement to split this string into two and return the following output:
\begin{verbatim}
[[1]]
[1] "Grand"

[[2]]
[1] "Opening"
\end{verbatim}

hint:  use strsplit() function
\subsubsection{Answer}
\label{sec:orgdbc3f9e}
A very inefficient solution\ldots{}
\begin{verbatim}
> list(sapply(strsplit(string, " "), "[", 1), sapply(strsplit(string, " "), "[", 2))
[[1]]
[1] "Grand"

[[2]]
[1] "Opening"
\end{verbatim}

\subsection{Exercise 10 (Bonus)}
\label{sec:orgd7027b4}
\subsubsection{Question}
\label{sec:org049e7ac}
Let
\begin{verbatim}
> y <- list("a", "b", "c")
> q <- list("A", "B", "C", "a", "b", "c").
\end{verbatim}

Write an R statement that will return all elements of q that are not in y, with the following result:
\begin{verbatim}
[[1]]
[1] "A"

[[2]]
[1] "B"

[[3]]
[1] "C"
\end{verbatim}

hint:  use setdiff() function
\subsubsection{Answer}
\label{sec:org4e0fc32}
\begin{verbatim}
> setdiff(q, y)
[[1]]
[1] "A"

[[2]]
[1] "B"

[[3]]
[1] "C"
\end{verbatim}
\end{document}
